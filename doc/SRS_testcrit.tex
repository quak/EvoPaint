\documentclass[titlepage,12pt]{scrartcl}
\usepackage{times,t1enc,graphicx,mathptmx,listings,exscale}
\usepackage[utf8]{inputenc}

\parindent0pt
\parskip1.5ex
\renewcommand{\baselinestretch}{1.3}

\begin{document}

\title{Testresults - SRS Test}
\author{Augustin Malle}

\subject{EvoPaint}

\maketitle

\thispagestyle{empty}
\tableofcontents

\newpage

\section{Testing EvoPaint}

\subsection{Purpose}
The purpose of this document is to record the testing results for EvoPaint. Abnormalities will be recorded and documented. This document will serve developers as a error sheet.

\subsection{Test Requirements}



\subsubsection{Painting}

\paragraph{1.2.1.1 Introduction/Purpose of feature}
With the painting tool EvoPaint provides an action which can be used to directly take influence onto the evolution. Within this tool the color and the rules for the painted section are adjustable. Also the size of the brush can be changed. There are different painting options. The user mainly paints color and rules at the same time. There also is the option to turn off one of these two to for example just paint the preset color. In this case the rules which are already present at the evolutions screen are saved and just the color of the regarding pictures are changed. On the other side it is also possible just to paint the rules. In this case the colors of the pixels are kept and the rules would be overwritten. The user should also have the possibility to paint into the evolution screen without using any rules.
There should also be a fairy dust option. In the painting case this fills the area beneath the brush with pixel of random color.

\paragraph{1.2.1.2 Stimulus/Response sequence}
By clicking on the paint brush icon located in the toolbox the paint tool is activated.
The user is now able to paint with the mouse (pressed left mouse button) on the evolution screen.

\paragraph{1.2.1.3 Testing results}
\begin{itemize}
\item Color adjustable			:	requirement ok
\item Rules adjustable			: requirement ok
\item Fairy dust						:	requirement ok
\item Color without rules		:	requirement ok
\item Rules without color		:	requirement ok
\item Brush size adjustable	:	requirement ok
\item Paint rules and color		:	requirement ok
\item Paint nothing					: requirement ok
\end{itemize}





\subsubsection{Move evolutions}

\paragraph{1.2.2.1 Introduction/Purpose of feature}
The move tool provides the user the possibility to move the evolution screen. So it is possible to focus interesting sections of the evolution. It also provides the user a full view of the evolutionary system without restrictions.

\paragraph{1.2.2.2 Stimulus/Response sequence}
The move tool is activating by clicking on the move icon in the tool menu.

\paragraph{1.2.2.3 Testing results}
\begin{itemize}
	\item Moving the evolution 	: requirement ok
\end{itemize}






\subsubsection{Selecting evolution parts}
\paragraph{1.2.3.1 Introduction/Purpose of feature}
The feature enables the user to select different evolution parts. This feature is the basic for other functions like "Fill Selection" or "Copy Seleciton".
\paragraph{1.2.3.2 Stimulus/Response sequence}
The user is able to select a part of the evolution using the select tool located in the toolbox.
\paragraph{1.2.3.3 Testing results}
\begin{itemize}
	\item Selecting parts of the evolution 	: broken border not painted 
\end{itemize}

\subsubsection{Rule eidtor}
\paragraph{1.2.4.1 Introduction/Purpose of feature}
With the rule editor the user can create and edit rules. After this the rules can be applied to the different pixels / organisms. This rule editor needs to provide a user friendly interface to define boolean expressions which the user can intuitively handle.
\paragraph{1.2.4.2 Stimulus/Response sequence}
The rule editor is accessible in the rule paint menu.
\paragraph{1.2.4.3 Testing results}
\begin{itemize}
	\item new rule set collection 						: requirement ok
	\item new rule set												: requirement ok
	\item delete rule set collection					: requirement ok
	\item delete rule set											:	requirement ok
	\item edit rule set c. description				: requirement ok
	\item edit rule set description						:	requirement ok
	\item copy rule set												:	requirement ok
	\item copy rule set collection 						:	requirement ok
	\item export rule set to clip board 			: requirement ok
	\item import rule set 										: requirement ok
	\item create new boolean rules						: bugged ( make new rule - do not define a target - if the user press ok an error dialog appears if he press cancel he is able to use the rule)
	\item create additional boolean condition	:	requirement ok
	\item add boolean target condition				:	requirement ok
	\item delete additional boolean condition	:	requirement ok
	\item delete boolean target condition			:	requirement ok
	\item select concerning target						:	requirement ok
	\item select condition target							:	requirement ok
	\item recognise ident conditions					:	requirement ok
	\item select actions
	\subitem change energy - (requirement ok)
	\subitem copy - (requirement ok)
	\subitem move - (requirement ok)
	\subitem assimilate - (requirement ok)
	\subitem proceed with partner - (requirement ok)
	\subitem set color - (requirement ok)
	\item select condition
	\subitem existence - (requirement ok)
	\subitem energy - (requirement ok)
	\subitem color likeness, me - (requirement ok)
	\subitem color likeness, color - (requirement ok)
	\item select targets											:	requirement ok
	\item select comparison										:	requirement ok
	\item select color												:	requirement ok
	\item select dimensions										:	requirement ok	
	\item select likeness in percent					:	requirement ok
	\item adjust values												:	requirement ok		
	\item select neightbors button						:	requirement ok	
	
	
	
\end{itemize}

\subsubsection{Zoom}
\paragraph{1.2.5.1 Introduction/Purpose of feature}
This tool enables the user to zoom into the evolution and focus on tiny details.
\paragraph{1.2.5.2 Stimulus/Response sequence}
The user can zoom into the evolution with simply scrolling his mouse wheel.
\paragraph{1.2.5.3 Testing results}
\begin{itemize}
	\item Zooming															: requirement ok
\end{itemize}

\subsubsection{Stop evolution}
\paragraph{1.2.6.1 Introduction/Purpose of feature}
With this feature the current evolution screen should be painted all together. That means that the cpu is heavily disburdened.
\paragraph{1.2.6.2 Stimulus/Response sequence}
To make use of this tool the user just needs to press the stop button located at the top of the toolbox. The evolution is stopped immediately.
\paragraph{1.2.6.3 Testing results}
\begin{itemize}
	\item Stop the evolution 	: requirement ok
\end{itemize}

\subsubsection{New evolution}
\paragraph{1.2.7.1 Introduction/Purpose of feature}
This feature should totally refresh the program. The size of the new Evolution should be defined by the user. The old evolution should be totally dropped.
\paragraph{1.2.7.2 Stimulus/Response sequence}
The "New evolution" feature can be accessed over the menu bar (File -> new evolution). In the following panel the user needs to insert the width and the high of the new evolution which should be created. The saved rules should stay unaffected.
\paragraph{1.2.7.3 Testing results}
\begin{itemize}
	\item Creating a new evolution 	: broken
\end{itemize}

\subsubsection{Open evolution}
\paragraph{1.2.8.1 Introduction/Purpose of feature}
The "load evolution" feature provides the user to load an evolution from a saved EvoPaint file.
\paragraph{1.2.8.2 Stimulus/Response sequence}
This feature is accessible over the menu bar (World'Open). Thereupon appears a dialog where the user can choose the EvoPaint-file which he wants to load in the current evolution screen.
\paragraph{1.2.8.3 Testing results}
\begin{itemize}
	\item Open evolutions 	: broken
\end{itemize}

\subsubsection{Save evolution}
\paragraph{1.2.9.1 Introduction/Purpose of feature}
This function saves a current evolution from the evolution screen. This gives the user the possibility to save interesting evolutionary systems for later consideration.
\paragraph{1.2.9.2 Stimulus/Response sequence}
The feature is accessible over the menu bar (World'Save, World'Save as). The difference between these two possibilities is that the first one does not provide a dialog in which the user can determine where he can save a file if the evolution was saved before.
\paragraph{1.2.9.3 Testing results}
\begin{itemize}
	\item Save evolution 	: broken
\end{itemize}

\subsubsection{Import}
\paragraph{1.2.10.1 Introduction/Purpose of feature}
With this feature it is possible to import any pictures from the file directory into EvoPaint. Of course there are no rules which could be imported. 
\paragraph{1.2.10.2 Stimulus/Response sequence}
This feature is accessible over the menu bar (World'Import). Thereupon a dialog is shown where the user can choose the pictures to be imported. The user now can choose the desired file and press ok. Now the picture will be loaded into the evolution screen and it can be used for further editing.
\paragraph{1.2.10.3 Testing results}
\begin{itemize}
	\item Importing a picture 	: requirement ok
\end{itemize}

\subsubsection{Export}
\paragraph{1.2.11.1 Introduction/Purpose of feature}
With this function the user can save pictures of the current evolution screen. They can be saved in the jpg or the png file format.
\paragraph{1.2.11.2 Stimulus/Response sequence}
The export function is accessible over the menu bar (World' Export). If the user clicks on the option "Export" a java save dialog is shown. There the user can choose the file format either png or jpg, the place the file is to be saved and the name of the file to be saved. By clicking on the "Ok" button of the java save dialog the picture is saved at the specified place. By clicking on the "Cancel" button the action is cancelled and EvoPaint application and evolution is continuing.
\paragraph{1.2.11.3 Testing results}
\begin{itemize}
	\item Exporting pictures 	: broken (after stopping the evolution and exporting the picture the evolution proceeds)
\end{itemize}

\subsubsection{End}
\paragraph{1.2.12.1 Introduction/Purpose of feature}
This function stops the evolution and the simply close the EvoPaint.
\paragraph{1.2.12.2 Stimulus/Response sequence}
The function is accessible over the menu bar (World'End). Thereupon a dialog appears where the user is asked if he really wants to end EvoPaint. By clicking on "Ja" the program exits. By clicking on the second option "Nein" the dialog is closed without further reaction.
\paragraph{1.2.12.3 Testing results}
\begin{itemize}
	\item Shut down EvoPaint 	: requirement ok
\end{itemize}

\subsubsection{Set name of a selection}
\paragraph{1.2.13.1 Introduction/Purpose of feature}
The purpose of this function is to order and keep the overview to the different selections created by the selection tool. So the user has the possibility give these selections a specific name.
\paragraph{1.2.13.2 Stimulus/Response sequence}
This function is accessible over the menu bar (Selections'Set Name). By clicking on this option a dialog appears where the user can enter the desired name.
\paragraph{1.2.13.3 Testing results}
\begin{itemize}
	\item Set name of a selection 	: requirement ok
\end{itemize}

\subsubsection{Fill}
\paragraph{1.2.14.1 Introduction/Purpose of feature}
This function fills a selection with a color. The user can choose the color and click on the option fill in the menu bar.
\paragraph{1.2.14.2 Stimulus/Response sequence}
The color needs to be set within the paint options at the right side beneath the toolbox. By clicking on the fill options (Selection ' Fill) the selection is filled with the desired color
\paragraph{1.2.14.3 Testing results}
\begin{itemize}
	\item Filling the screen with rules : requirement ok
	\item Filling the screen with color	: requirement ok
	\item Filling the screen with rules and color	: requirement ok
	\item Filling a selection : requirement ok
\end{itemize}

\subsubsection{Fill 50}
\paragraph{1.2.15.1 Introduction/Purpose of feature}
This function fills a selection with a color to 50. The user can choose the color and click on the option fill 50 in the menu bar.
\paragraph{1.2.5.2 Stimulus/Response sequence}
Like at the fill option the color needs to be set in the paint options beneath the toolbox. Activating this function fills the selection with the preset color
\paragraph{1.2.15.3 Testing results}
\begin{itemize}
	\item Fill with 50 percent color :	todo daniel
\end{itemize}

\subsubsection{Open as new}
\paragraph{1.2.16.1 Introduction/Purpose of feature}
This feature opens a selection in a new EvoPaint evolution screen. The purpose of this feature is to extract a part of an evolution and observe it out of the source system
\paragraph{1.2.16.2 Stimulus/Response sequence}
The feature is accessible over the menu bar (Selection ' Open as new). A new EvoPaint Window with the chosen selection is opened
\paragraph{1.2.16.3 Testing results}
\begin{itemize}
	\item Open as new : broken
\end{itemize}

\subsubsection{Copy}
\paragraph{1.2.17.1 Introduction/Purpose of feature}
With this function the user can copy different Selections and paste them on other places at the evolution screen.
\paragraph{1.2.17.3 Testing results}
\begin{itemize}
	\item copy selection 	: broken do not copy rules
\end{itemize}

\subsubsection{Delete current}
\paragraph{1.2.18.1 Introduction/Purpose of feature}
With this function the user is able to delete previous selections.
\paragraph{1.2.18.2 Stimulus/Response sequence}
The user needs to select the selection he wants to delete and click on the option delete current accessible in the menu bar (Selections' Delete current).
\paragraph{1.2.18.3 Testing results}
\begin{itemize}
	\item delete current selection : requirement ok
\end{itemize}

\subsubsection{Select selections}
\paragraph{1.2.19.1 Introduction/Purpose of feature}
With this feature the user is able to select different selection which he created between the evolution time of EvoPaint.
\paragraph{1.2.19.2 Stimulus/Response sequence}
This feature is accessible within the menu bar. (Selection ' Selection).
\paragraph{1.2.19.3 Testing results}
\begin{itemize}
	\item Selecting selections 	: requirement ok
\end{itemize}

\subsubsection{Clear selections}
\paragraph{1.2.20.1 Introduction/Purpose of feature}
This feature enables the user to delete all selection. This cannot be undone.
\paragraph{1.2.20.2 Stimulus/Response sequence}
This feature is accessible within the menu bar. (Selection ' Clear Selections)
\paragraph{1.2.20.3 Testing results}
\begin{itemize}
	\item Clear selections 	: requirement ok
\end{itemize}

\subsubsection{Erase tool}
\paragraph{1.2.21.1 Introduction/Purpose of feature}
With this tool its possible to erase areas of the evolution. The affected areas will loose all asserted pixels and their rules. 
\paragraph{1.2.21.2 Stimulus/Response sequence}
This feature is accessible within the tool bar. 
\paragraph{1.2.21.3 Testing results}
\begin{itemize}
	\item Erase tool 	: requirement ok
\end{itemize}

\subsubsection{Pick tool}
\paragraph{1.2.22.1 Introduction/Purpose of feature}
With this tool its possible to pick up the color and the rule of a pixel. 
\paragraph{1.2.22.2 Stimulus/Response sequence}
This feature is accessible within the tool bar. 
\paragraph{1.2.22.3 Testing results}
\begin{itemize}
	\item Pick tool 	: requirement ok/no rules are picked
\end{itemize}


\subsubsection{Agent Simulation}
\paragraph{1.2.23.1 Introduction/Purpose of feature}
This feature enables the "Agent Simulation" mode for EvoPaint. 
\paragraph{1.2.23.2 Stimulus/Response sequence}
This feature is accessible on the right top of the evolutionscreen. 
\paragraph{1.2.23.3 Testing results}
\begin{itemize}
	\item Agend mode	: requirement ok
\end{itemize}


\subsubsection{Cellular Automat}
\paragraph{1.2.24.1 Introduction/Purpose of feature}
This feature enables the "Cellular Simulation" mode for EvoPaint.
\paragraph{1.2.24.2 Stimulus/Response sequence}
This feature is accessible on the right top of the evolutionscreen.
\paragraph{1.2.24.3 Testing results}
\begin{itemize}
	\item Cellular mode 	: requirement ok
\end{itemize}


\subsubsection{Record Video}
\paragraph{1.2.25.1 Introduction/Purpose of feature}
With this feature the user is able to record videos of different evolutions. Most of the time it is quit hard to exactly reconstruct a previous and interesting evolution. In these cases the user can record the evolution for later validations.
\paragraph{1.2.25.2 Stimulus/Response sequence}
This tool is accessible within the "Control Panel". 
\paragraph{1.2.25.3 Testing results}
\begin{itemize}
	\item Recording a video	: requirement ok
\end{itemize}


\subsubsection{Reset EvoPaint}
\paragraph{1.2.26.1 Introduction/Purpose of feature}
This feature resets EvoPaint and its evolution screen.
\paragraph{1.2.26.2 Stimulus/Response sequence}
This tool is accessible within the "Control Panel". 
\paragraph{1.2.26.3 Testing results}
\begin{itemize}
	\item Reset EvoPaint	: broken selections bleiben erhalten
\end{itemize}


\subsubsection{Pause EvoPaint}
\paragraph{1.2.27.1 Introduction/Purpose of feature}
This pauses the evolution
\paragraph{1.2.27.2 Stimulus/Response sequence}
This tool is accessible within the "Control Panel". 
\paragraph{1.2.27.3 Testing results}
\begin{itemize}
	\item Pause evopaint 	: requirement ok
\end{itemize}


\end{document}
	